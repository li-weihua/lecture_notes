\chapter{Calculus of Variations}

\section{Euler-Lagrange Equation}

\begin{lemma}
\label{lemma:variation-1}
If $f(x) \in C[a,b]$, and if
\begin{equation*}
\int_a^b f(x) g(x) \,dx = 0
\end{equation*}
for $\forall g(x) \in L(a,b)$ such that $g(a) = g(b) = 0$, 
then $f(x) = 0, \forall x \in [a, b]$.
\end{lemma}

\begin{proof}
Suppose $f(x) > 0$ in some $[x_1, x_2]$, if we set $g(x) = (x - x_1)(x_2 - x)$, then
\begin{equation*}
\int_{x_1}^{x_2} f(x)(x - x_1)(x_2 - x)\,dx > 0 \,,
\end{equation*}

since the integrand is positive. This contradiction proves the lemma.
\end{proof}

\begin{remark}
The \autoref{lemma:variation-1} still holds if we replace linear space $L(a,b)$ 
by normed linear space $N_n(a,b)$. To proof this, just set
\begin{equation*}
    g(x) = \left[ (x - x_1)(x_2 - x) \right]^n
\end{equation*}
\end{remark}


Problem: find the extremum of \cref{eq:variation-1}.

\begin{equation}
    \label{eq:variation-1}
    J[y] = \int_{a}^{b} F[x, y, y']\, dx
\end{equation}


Solution 1:

\begin{equation*}
\begin{split}
0 & = \delta J[y] \\
  & = \int_a^b \left[F(x, y + \delta y, y' + \delta y') - F(x, y, y')\right] \, dx\\ 
  & = \int_a^b \left[\frac{\partial F}{\partial y} \delta y 
      + \frac{\partial F}{\partial y'} \delta y' \right]\,dx\\
  & = \int_a^b \left[ \frac{\partial F}{\partial y} \delta y
      - \frac{d}{dx} \left( \frac{\partial F}{\partial y'}\right) \delta y \right] dx 
      + \left.\frac{\partial F}{\partial y'} \delta y \right\vert_a^b \\
  & = \int_a^b \left[ \frac{\partial F}{\partial y}
      - \frac{d}{dx} \left( \frac{\partial F}{\partial y'}\right) \right]\delta y \,dx 
\end{split}
\end{equation*}

So, 
\begin{equation}
\label{Euler-Lagrange-equation}
\frac{\partial F}{\partial y} - \frac{d}{dx} \left( \frac{\partial F}{\partial y'}\right) = 0
\end{equation}
which is called Euler-Lagrange equation.